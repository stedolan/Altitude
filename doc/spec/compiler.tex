\chapter{Compiler}
\section{CIL}
To perform the work of parsing and typechecking, we are using the CIL (C Intermediate Language) library, from Berkeley University. It parses C source and does some transformation to reduce the set of constructs down to a manageable amount, by introduction of temporary variables, etc.

For instance, after CIL transforms the source, each assignment has as its right-hand-side either a single function call, or a side-effect-free expression. Complex expressions such as {\tt a = f() + g()} are transformed down to this form by the introduction of temporaries.

\subsection{Merging and Linking}

One task of a C implementation is to link different parts of a program by resolving external symbols. For instance, suppose a function is declared (given a prototype) in several files (usually by means of a header file), and defined once. All calls to the function must run the same code wherever it's defined, and all pointers to that function must compare equal.

This can be done internally by CIL, which calls it ``merging'', and was written to facilitate whole-program analysis by CIL's authors.

\subsection{Preprocessing}
Before C code is run, it must be preprocessed. This is a task we're not going to address. Running C on Altitude will require an external preprocessor. This isn't a big deal, many standalone preprocessors exist. For now, we shall use the GNU preprocessor as packaged with GCC.

\section{S-Expression Intermediate Representation}

CIL is written in OCaml. To pass the CIL representation of the program to C, there is an OCaml module to dump the CIL in an s-expression based tree format.

\section{S-Expression Parsing and Compilation}
